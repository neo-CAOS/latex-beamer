% 02-qemu-emulation.tex

\section{QEMU Board Emulation}

\begin{frame}{Custom QEMU Version}
    \begin{itemize}
        \item Emulate the NXP S32K3X8EVB board, which is not natively supported by QEMU.
        \item Ensure proper emulation of the CPU (ARM Cortex-M7), memory map, and peripherals.
    \end{itemize}
\end{frame}

\begin{frame}{Technical Details}
    \begin{itemize}
        \item Added a new machine model to QEMU for the S32K3X8EVB board.
        \item Used previous QEMU implementations as reference (ARMv7-M architecture).
        \item Implemented custom initialization routines for memory and peripherals.
        \item S32K358EVB board has the following specification:
            \begin{itemize}
                \item ARM Cortex-M7 CPU.
                \item \(\sim \)8MB Flash memory, 768KB SRAM, 256KB DTCM, and 128KB ITCM.
                \item NVIC with 256 IRQs and 4 priority bits.
                \item Multiple peripherals: 16 LPUART, 3 PIT timers, 16 MPU regions.
                \item System clock running at 240MHz.
            \end{itemize}
    \end{itemize}
\end{frame}

\begin{frame}{Memory Regions Initialization}
    \begin{itemize}
        \item \textbf{Flash Memory}: Configured multiple blocks:
        \begin{itemize}
            \item Block0: Base Address: 0x00400000, Size: 2 MB
            \item Block1: Base Address: 0x00600000, Size: 2 MB
            \item Block2: Base Address: 0x00800000, Size: 2 MB
            \item Block3: Base Address: 0x00A00000, Size: 2 MB
            \item Block4: Base Address: 0x10000000, Size: 128 KB
            \item Utest: Base Address: 0x18000000, Size: 8 KB
        \end{itemize}
        \item \textbf{SRAM Memory}:
        \begin{itemize}
            \item Block0: Base Address: 0x20400000, Size: 256 KB
            \item Block1: Base Address: 0x20440000, Size: 256 KB
            \item Block2: Base Address: 0x20480000, Size: 256 KB
        \end{itemize}
        \item \textbf{DTCM and ITCM Memory}:
        \begin{itemize}
            \item DTCM0: Base Address: 0x20000000, Size: 128 KB
            \item ITCM0: Base Address: 0x00000000, Size: 64 KB
        \end{itemize}
    \end{itemize}
\end{frame}

\begin{frame}{Peripherals and Interrupts Setup}
    \begin{itemize}
        \item \textbf{NVIC (Nested Vectored Interrupt Controller)}:
        \begin{itemize}
            \item Configured with 256 IRQs and 4 priority bits.
            \item Interrupt vectors mapped in system memory.
        \end{itemize}
        \item \textbf{LPUART (Low Power UART)}:
        \begin{itemize}
            \item Base Address: 0x4006A000
            \item The board has \textbf{16 LPUART instances}.\\ They are mapped starting from the UART base address. 
            \item Connected to NVIC and clocked by \texttt{AIPS\_PLAT\_CLK} and \texttt{AIPS\_SLOW\_CLK}.
        \end{itemize}
        \item \textbf{PIT Timers (Periodic Interrupt Timer)}:
        \begin{itemize}
            \item Timer1: Base Address: 0x40037000
            \item Timer2: Base Address: 0x40038000
            \item Timer3: Base Address: 0x40039000
        \end{itemize}
        \item \textbf{MPU}: 16 regions.
    \end{itemize}
\end{frame}

\begin{frame}{System Clocks and Interrupts}
    \begin{itemize}
        \item \textbf{System Clock}:
        \begin{itemize}
            \item Primary System Clock: 240MHz frequency, 4.16ns period.
            \item AIPS Platform Clock: 80MHz
            \item AIPS Slow Clock: 40MHz
            %\item Reference Clock: 1MHz
        \end{itemize}
        \item \textbf{Interrupt Handling}:
        \begin{itemize}
            \item Configured NVIC to handle exceptions and IRQs.
            \item Supports up to 256 IRQs with 4 priority levels.
            \item NVIC is connected to system and reference clocks.
            %\item Bit-band support enabled for efficient memory access.
            \item Interrupt sources include timers, UARTs, and peripheral events.
        \end{itemize}
    \end{itemize}
\end{frame}

\begin{frame}{Firmware Loading}
    \begin{itemize}
        \item Function: \texttt{s32k3x8\_load\_firmware}
        \item Parameters:
        \begin{itemize}
            \item \texttt{cpu}: The ARM CPU instance.
            \item \texttt{ms}: The machine state.
            \item \texttt{flash}: The memory region representing the flash memory.
            \item \texttt{firmware\_filename}: The filename of the firmware to be loaded.
        \end{itemize}
        \item Functionality:
        \begin{itemize}
            \item Reads the firmware file and loads its contents into the specified flash memory region.
        \end{itemize}
    \end{itemize}
\end{frame}

\begin{frame}[fragile]{Class initialization}
    \begin{itemize}
    \item \texttt{s32k3x8\_class\_init}:
        \begin{lstlisting}[language=C]
static void s32k3x8_class_init(ObjectClass *oc, void *data) {
    MachineClass *mc = MACHINE_CLASS(oc);
    mc->name = g_strdup("s32k3x8evb");
    mc->desc = "NXP S32K3X8 EVB (Cortex-M7)";
    mc->init = s32k3x8_init;
    mc->default_cpu_type = ARM_CPU_TYPE_NAME("cortex-m7");
    mc->default_cpus = 1;
    mc->min_cpus = mc->default_cpus;
    mc->max_cpus = mc->default_cpus;
    mc->no_floppy = 1;
    mc->no_cdrom = 1;
    mc->no_parallel = 1;
}
        \end{lstlisting}
    \end{itemize}

\end{frame}


